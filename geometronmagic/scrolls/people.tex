\hypertarget{home}{%
\subsection{\texorpdfstring{\href{scrolls/home}{HOME}}{HOME}}\label{home}}

\hypertarget{people}{%
\section{People}\label{people}}

The heart of a network of people is the \emph{people}, not the
technology which connects us. We aim to build a network which can
materially support creators on the network. To do this we must make the
network provide so much more benefit to participants than the cost of
adding physical infrastructure and providing support to the creators
that people will do so out of self-interest.

We aim to build a network which helps as many people as much as
possible. We aim to start by helping the most marginalized people with
direct assistance in the form of communication, knowledge and power
resources. From there we aim to connect all the other people who share
physical space in local communities.

Big Tech doesn't care about anyone's community. They provide just enough
increases in efficiency and human connection between people to get us
addicted to their products, then extract all excess value that they can
using the power of monopolies to create a vast suction of value from
everyone everywhere in the world to the professionals of their industry.

Multiplied by hundreds of millions of communities, this leaves trillions
of dollars on the table in commerce which can be recaptured by local
communities away from Big Tech. If we can provide more increases in
efficiency and more human connection than Big Tech but keep the value in
our local communities, we should be able to take every single dollar out
of their system, to totally destroy Silicon Valley. Our long term aim is
to purge the Earth of them, to totally destroy the culture and society
of the so-called ``tech industry''. We should be able to engage in
commerce anywhere in the world without sending a single dollar to
California.

This all starts with the simple act of creating networks of people
dedicated to the shared goal of building out this network.

The physical machines and digital files which make up the technology
system are released by the creators of the network from the property
system in order to make replication more free.

This chapter is a recruitment request. We are looking for all these
people to join us, to join this movement and this network. We need you
to collaborate to share knowledge and resources to help build this.
Ultimately we want this network to belong to everyone everywhere, but
right now we are directly asking all the people below to reach out and
collaborate to immediately bring this network into being in the physical
world.

\textbf{The people of the network.} We create this network to support
ourselves. We believe that building and releasing for free these
resources can create enough value for enough people that they will
support us directly to keep freely creating and sharing. It is my
intention to live this way as soon as possible, to get food, shelter,
medicine, transport, technical resources, and places to work provided by
people who benefit greatly from the expansion of the network. This will
start out as a very hard thing, supported by a few large donors and many
small donors, and will become easy as we get network scale effects. The
author will become a person of the network as soon as I possibly can.

\textbf{Public Librarians.} The Pibrary is an extension of the library
system. The public library mission is make knowledge as free as
possible. The pibrary consists of freely created, edited and shared
documents on free hardware maintained in public spaces and available to
all. Public libraries currently serve as a computing and Internet
resource for people with no other access. The Pibrary brings physical
computers out to the streets with 24/7 access, taking some load off of
the computing resources in the physical library, and extending hours and
accessibility of the resource. By being useful for other communities,
the Pibrary can be self-supporting, extending the reach and impact of
the library without costing money from the base library budget.
Pibraries also extend the effective collection of the library, as
user-generated content is all completely free and openly shared,
replicated globally from one to another potentially over billions of
servers. As we scale up, it should be possible for a local public
library with almost no budget or resources to have access to a very
large collection of free titles. Libraries with maker spaces already
have STEM education curriculum and often already have Raspberry pi's,
and this network simply adds to the impact of that, adding content to
it.

\textbf{Authors.} If you are someone who writes for the common good of
humanity who believes in free knowledge without the limits of
intellectual property, the Pibrary can be a platform for creating
explicitly free work. As we scale, the intent of this network is to have
authors created through direct material support from the communities we
serve. If we can get authors generating knowledge to provide all the
good things in life, this will be a self-sustaining system without
money. In the short term, it can sustain authors via driving traffic to
voluntary support pages like Patreon or links to buy physical copies of
books with a profit built in from self-publishing platforms like Lulu
press. No work may be posted unless you relinquish all copyright to
public domain. No exceptions.

\textbf{Unhoused people.} This is a mutual aid based network. This means
primarily that we are providing fully free infrastructure and services
in public spaces on the streets, directly controlled by the people who
need it. This means we will provide free wifi hotspots, free solar
powered device chargers, and free easily portable physical computers you
can use 24/7. It also means that we will be co-creating documents which
have all the resources available in the local community including
contact information for aid organizations, jobs, housing resources, harm
reduction organizations, and anyone else who is providing resources. It
also means we are going to be asking for your help in sharing and
growing the network. If our network grows we will eventually be able to
make all the things of a good life free for all people, and the only way
to do that is to provide for those who have the least first. And the way
to do that is to incorporate you directly into our emerging community,
where you can actively engage and contribute by sharing your own stories
and creations with the other readers and creators on the Pibrary
network.

\textbf{Teachers.} This system can be used by writing teachers to help
students co-create published books directly to their local community as
an alternative to papers read only by the teacher. It can also be used
by anyone teaching people how to code as a development environment which
can be run in a web browser, then published to the network and shared
with the world instantly without any gate keepers. Rather than learning
how to code in a job for a company, we teach people to code by directly
building web content immediately and publishing to the network for other
people to build on. Our aim is to transition all teaching from job
training to direct creation of useful resources for people's
communities. The Raspberry Pi can save a lot of money as a low cost
powerful computing resource, and the more people we share this with the
better.

\textbf{Traveling kids/dirty kids/crust punks.} Think of the Pibrary as
a deep sign, or perhaps ``hypersign''. A cardboard sign, cloth flag with
url or QR code can point to a web-based mirror of the chaos books. These
books can have \emph{anything} on them. You can share your stories,
share whatever you have to share, to sell, etc. By forming connections
with other people who maintain the networks and are supportive, you can
maintain documents for free online with no gate keepers, no censors, no
algorithms, no passwords or logins, just free things to share freely
which can help get the information out there that you need to get the
aid you need to stay happy and healthy on the road. You have a critical
role to play in growing the network, sharing our resources and story
with the world. This is a knowledge network, a linking of people who
inhabit shared physical spaces. You can help us to link up all the
social networks which connect in physical crossroads like downtowns and
truck stops across the world. And we can help you by helping to promote
direct mutual aid to help you on the road. In a world without property,
you are also the pumping blood of our network, moving physical goods to
place to place without money.

\textbf{Scientists, mathematicians, academics.} This is a publication
platform with no barrier to entry. If you produce knowledge you are
willing to share freely with humanity, this can be a platform which not
only shares what you have created but which is built in such a way that
others can immediately build on it. You can write a totally incomplete
paper with most of the important parts missing, and if it gets to the
correct collaborators, they can build on it and replicate it back to the
network and you'll get back something much better than what you started
with. This is a new way of doing research, where we do not associate
documents with individual people but with a process of improvement where
all readers and writers are co-creating the work over time. We advocate
letting go of ego and prioritizing progress over personal accolades.
However, this publication is still compatible with career boosting
publications as it can be treated like free online archives are now with
preprints of articles that eventually go into gate kept peer review
journals. These documents are compatible with the LaTeX system of
mathematical type setting, a small modification is all that is required
to turn it on(documented here).

\textbf{Keepers of indigenous knowledge.} We aim with this knowledge
network to bring back a more dynamic living type of knowledge that has
existed throughout the world in indigenous cultures for thousands of
years. It is our hope that by bringing free computers, free Internet and
free off grid power for it all to communities with posses indigenous
knowledge that those people will be empowered to share using this
platform, both with each other and with the world. Our network is a
hybrid of oral tradition and digital methods, where community members
are all co-creating documents which are then passed along freely to the
whole of the community. The world today is in desperate urgent need of
indigenous science and technology if we are to restore equilibrium
between humanity and the living world. We need the traditional
technology and culture to be able to blend with that of the Internet and
computers if we are to navigate the whole of humanity out of our current
predicament. Our intent is to get the hardware into your communities,
teach your teachers, elders, and other stake holders how to run and grow
the system, and then it becomes your network to shape as you see fit. We
also aim to have the network of off-grid computers and wireless links
connect with environmental sensors, putting the living Earth onto the
indigenous network in a very literal way, hopefully giving it more of a
voice in the affairs of the our world as well.

\textbf{Mutual aid workers, harm reduction, street outreach, community
organizers.} The pibrary should have a book dedicated to community
resources. This can be a directory of links to resources, people,
places, organizations, jobs, housing, really anything that is freely
available to help people out should get cataloged here and that should
be actively maintained by all. Think of this like a phone book for
resources for those most in need of those resources(any resources).

\textbf{People who have too many physical books.} You know who you are.
You know more than half your books are ones you'll never look at again
and don't need but both are not sure which half that is, can't bear the
thought of them going into a dumpster due to a library donation drive
getting too many books and don't want to lug them all over the place.
What you really want is to get them into the hands of an actual reader
who will actually read them. The pibrary will have book catalogs of
physical books which people are willing to give away(not loan!!!). List
whatever books you want, along with either contact info or contact info
of someone willing to manage the network of the book exchange. Then if
someone wants a book you have they can ask and you can just pass it
along with either a meeting or a drop off at a common location like a
coffee shop or public library.

\textbf{Artists.} The art you sell does not have to be free, but the
media describing it does. This is a platform on which artists can
co-create whole books which catalog the art they create and sell or
promote whatever commercial channels they use for that. This is not an
advertising platform. Spam gets deleted. But it is a place where people
can create long form exposition of whatever they produce and place all
that in the context of other creators' art and craft products.

\textbf{Deep readers.} The knowledge which can only be attained through
reading a lot of books is of great value to a library community. We need
people to put together libraries, to organize content, to edit, and to
add manuscripts which are already available for free but not widely
distributed. We also need people to curate libraries, to figure out
exactly what people can benefit from in a given community. That can only
happen with very active participation by people who read a lot, both
widely and with some depth into various fields. Reading lists are of the
utmost importance.

\textbf{Practitioners of religion/magic/spirituality.} The books we
co-create with the Pibrary network are living documents. What better way
to transmit wisdom could you ask for? True wisdom does not belong to
anyone. True wisdom can withstand the maelstrom of a chaotic co-editing
process by potentially billions of readers and writers and end up better
than it started. We ask that you consider sharing what you know and what
you have learned from your teachers in this truly free form.

\textbf{Technology creators.} One of the core functions of this
knowledge network is to spread the technical knowledge required for
people to build a new civilization from the waste streams of the
existing one. This requires a whole new way of creating technology,
based on free sharing of knowledge over this network. Ideally this can
be a self-sustaining way to exist in society as a creator of technology.
We can create technology, share detailed documents on how to build it,
and it will come back to us with community additions much better than
what we built. As the network grows and we build more and more truly
free infrastructure (manufacturing, housing, food production, power,
etc.) we can eventually fully support ourselves off of this network. We
will release our creations for free into the network and get back more
free stuff than we put in as the network effect accelerates innovation.

\textbf{Organizers.} The chaos book represents an alternative method of
communication which can be very valuable for organizing. Rather than
``news'', a chaos book represents a body of knowledge which is bound by
a common theme or purpose, not a flow of new information all the time
but a reference work of use to all. You can create manuals for
organizing, directories of organizers, and compendiums of community
stories and issues by asking for input and compiling it all into chaos
books.

\textbf{Local government.} Most people don't even care that you exist.
This network of documents can be used to create community knowledge
which includes the activities of local government, increasing engagement
and effectiveness.

\textbf{Art Gallery Owners.} An art gallery is already a type of
library, as well as a network. It links artists with the public and with
patrons. It links art with the rest of the cultural context in which
that art was created. And finally it is a physical place, the survival
of which depends on how people view its meaning. Building a free book
which represents the meaning of the \emph{place} that is a gallery,
studio or other art space can act to amplify the value of that place,
benefiting the mission of the space.
