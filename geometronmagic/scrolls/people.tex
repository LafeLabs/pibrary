

We aim to build a network which helps as many people as much as
possible. We aim to start by helping the most marginalized people with
direct assistance in the form of communication, knowledge and power
resources. To support all this, we are building a network which takes
back the value generated by the Internet to local communities with much
simpler and more human connections based around common spaces. These
human connections provide value to different people in different ways,
and in this chapter we discuss different roles and benefits that people
can get from a local network. The ``tech'' will end up being incidental.
The architecture we are creating is one not of computers but of people.

Big Tech doesn't care about anyone's community. They provide just enough
increases in efficiency and human connection between people to get us
addicted to their products, then extract all excess value that they can
using the power of monopolies to create a vast suction of value from
everyone everywhere in the world to the professionals of their industry.

Multiplied by hundreds of millions of communities, this leaves trillions
of dollars on the table in commerce which can be recaptured by local
communities away from Big Tech. If we can provide more increases in
efficiency and more human connection than Big Tech but keep the value in
our local communities, we should be able to take every single dollar out
of their system, to totally destroy Silicon Valley. Our long term aim is
to purge the Earth of them, to totally destroy the culture and society
of the so-called ``tech industry''. We should be able to engage in
commerce anywhere in the world without sending a single dollar to
California.

This chapter is a recruitment request. We are looking for all these
people to join us, to join this movement and this network. We need you
to collaborate to share knowledge and resources to help build this.
Ultimately we want this network to belong to everyone everywhere, but
right now we are directly asking all the people below to reach out and
collaborate to immediately bring this network into being in the physical
world.

\textbf{The people of the network.} We create this network to support
ourselves. We learn to build all the parts of the network, and travel
from place to place teaching others to copy the whole system. Ultimately
we hope this category includes \emph{everyone}, since if this all scales
up all of humanity can live off of the flow of value across our network.
For now this is just anyone who knows the system, starting with me, the
author of this book. We curate knowledge into Pibrary, build technology
kits and give them away, build crafts and give them away, and teach
people to copy it all. We can work off of direct donations of food and
shelter and resources as well as large grants for community development
and education(discussed in a later chapter). Our primary mission is
replication: to teach others our ways sufficiently that they can become
a fully self-sufficient person of the network.

\textbf{Public Librarians.} The Pibrary is an extension of the library
system. The public library mission is make knowledge as free as
possible. The pibrary consists of freely created, edited and shared
documents on free hardware maintained in public spaces and available to
all. Public libraries currently serve as a computing and Internet
resource for people with no other access. The Pibrary brings physical
computers out to the streets with 24/7 access, taking some load off of
the computing resources in the physical library, and extending hours and
accessibility of the resource. By being useful for other communities,
the Pibrary can be self-supporting, extending the reach and impact of
the library without costing money from the base library budget.
Pibraries also extend the effective collection of the library, as
user-generated content is all completely free and openly shared,
replicated globally from one to another potentially over billions of
servers. As we scale up, it should be possible for a local public
library with almost no budget or resources to have access to a very
large collection of free titles. Libraries with maker spaces already
have STEM education curriculum and often already have Raspberry Pi's,
and this network simply adds to the impact of that, adding content to
it.

\textbf{Authors.} If you are someone who writes for the common good of
humanity who believes in free knowledge without the limits of
intellectual property, the Pibrary can be a platform for creating
explicitly free work. As we scale, the intent of this network is to have
authors created through direct material support from the communities we
serve. If we can get authors generating knowledge to provide all the
good things in life, this will be a self-sustaining system without
money. Thanks to print-on-demand publishing, it is possible for authors
to make money on hard copies of their Public Domain works, and that is
the most efficient way for us to make money on this. A book can be sold
for 15 dollars where an author makes 7, and people can buy it not for
themselves but to release for free into the community as a shared
artifact, like a library book. If we are only producing books which
people in a local community want to use widely, we can ask people with
some extra money to buy a number of copies and give them out. This is
the main way I intend to survive off of the network, by selling books.
No work may be posted unless you relinquish all copyright to public
domain.

\textbf{People who inhabit public spaces.} This includes anyone who does
not have access to indoor amenities due to the structural violence of
poverty. Ours is a mutual aid based network. This means primarily that
we are providing fully free infrastructure and services in public spaces
on the streets, directly controlled by the people who need it. This
means we will provide free wifi hotspots, free solar powered device
chargers, and free easily portable physical computers you can use 24/7.
It also means that we will be co-creating documents which have all the
resources available in the local community including contact information
for aid organizations, jobs, housing resources, harm reduction
organizations, and anyone else who is providing resources. It also means
we are going to be asking for your help in sharing and growing the
network. If our network grows we will eventually be able to make all the
things of a good life free for all people, and the only way to do that
is to provide for those who have the least first. And the way to do that
is to incorporate you directly into our emerging community, where you
can actively engage and contribute by sharing your own stories and
creations with the other readers and creators on the Pibrary network.
For our network to succeed, we must provide the most to those with the
least first, and those people must become active partners in our
venture.

\textbf{Teachers.} This system can be used by writing teachers to help
students co-create published books directly to their local community as
an alternative to papers read only by the teacher. It can also be used
by anyone teaching people how to code as a development environment which
can be run in a web browser, then published to the network and shared
with the world instantly without any gate keepers. Rather than learning
how to code in a job for a company, we teach people to code by directly
building web content immediately and publishing to the network for other
people to build on. Our aim is to transition all teaching from job
training to direct creation of useful resources for people's
communities. The Raspberry Pi can save a lot of money as a low cost
powerful computing resource, and the more people we share this with the
better.

\textbf{Traveling kids/dirty kids/crust punks.} Think of the Pibrary as
a deep sign, or perhaps ``hypersign''. A cardboard sign, cloth flag with
url or QR code can point to a web-based mirror of the chaos books. These
books can have \emph{anything} on them. You can share your stories,
share whatever you have to share, to sell, etc. By forming connections
with other people who maintain the networks and are supportive, you can
maintain documents for free online with no gate keepers, no censors, no
algorithms, no passwords or logins, just free things to share freely
which can help get the information out there that you need to get the
aid you need to stay happy and healthy on the road. You have a critical
role to play in growing the network, sharing our resources and story
with the world. This is a knowledge network, a linking of people who
inhabit shared physical spaces. You can help us to link up all the
social networks which connect in physical crossroads like downtowns and
truck stops across the world. And we can help you by helping to promote
direct mutual aid to help you on the road. In a world without property,
you are also the pumping blood of our network, moving physical goods to
place to place without money.

\textbf{Scientists, mathematicians, academics.} This is a publication
platform with no barrier to entry. If you produce knowledge you are
willing to share freely with humanity, this can be a platform which not
only shares what you have created but which is built in such a way that
others can immediately build on it. You can write a totally incomplete
paper with most of the important parts missing, and if it gets to the
correct collaborators, they can build on it and replicate it back to the
network and you'll get back something much better than what you started
with. This is a new way of doing research, where we do not associate
documents with individual people but with a process of improvement where
all readers and writers are co-creating the work over time. We advocate
letting go of ego and prioritizing progress over personal accolades.
However, this publication is still compatible with career-boosting
publications as it can be treated like free online archives are now with
preprints of articles that eventually go into gate kept peer review
journals. These documents are compatible with the LaTeX system of
mathematical type setting, a small modification is all that is required
to turn it on.

\textbf{Off grid experts.} Do you know how to work with solar panels, or
build small hydroelectric generators? Or how to can and pickle your own
food? How to build composting toilets? Organize a community garden? This
is your free media distribution channel. This platform is how we can get
your builds, your methods, your little tips and tricks out into the
physical world into the hands of the people who can use them. We will
help organize and curate a collection of your skills and methods into a
form which gets the absolute maximum impact. As stated earlier in this
work, the replication economy will provide a non-monetary return on your
investment, deploying the collective genius of millions or billions of
people to take what you have built and improve upon it exponentially and
bring back a replication of the evolved document with the better
technology.

\textbf{Keepers of indigenous knowledge.} We aim with this knowledge
network to bring back a more dynamic living type of knowledge that has
existed throughout the world in indigenous cultures for thousands of
years. It is our hope that by bringing free computers, free Internet and
free off-grid power for it all to communities with posses indigenous
knowledge that those people will be empowered to share using this
platform, both with each other and with the world. Our network is a
hybrid of oral tradition and digital methods, where community members
are all co-creating documents which are then passed along freely to the
whole of the community. The world today is in urgent need of indigenous
science and technology if we are to restore equilibrium between humanity
and the living world. The survival of our species now depends on our
ability to spread knowledge about how to live in equilibrium with an
ecosystem to all of humanity. We need the traditional technology and
culture to be able to blend with that of the Internet and computers if
we are to navigate the whole of humanity out of our current predicament.
Our intent is to get the hardware into your communities, teach your
teachers, elders, and other stake holders how to run and grow the
system, and then it becomes your network to shape as you see fit. We
also aim to have the network of off-grid computers and wireless links
connect with environmental sensors, putting the living Earth onto the
indigenous network in a very literal way, hopefully giving it more of a
voice in the affairs of the our world as well.

\textbf{Mutual aid workers, harm reduction, street outreach, community
organizers.}\\
We can train you to build and share the Pibrary system which will help
people charge their devices and get access to the Internet. Also, the
Pibrary will have a book dedicated to community resources which you can
both contribute to and share. This can be a directory of links to
resources, people, places, organizations, jobs, housing, really anything
that is freely available to help people out should get cataloged here
and that should be actively maintained by all. Think of this like a
phone book for resources for those most in need of those resources(any
resources).

\textbf{People who have too many physical books.} You know who you are.
You know more than half your books are ones you'll never look at again
and don't need but both are not sure which half that is, can't bear the
thought of them going into a dumpster due to a library donation drive
getting too many books and don't want to lug them all over the place.
What you really want is to get them into the hands of an actual reader
who will actually read them. We are building a network of places of
sharing, and all this centers on knowledge and books. You can use this
as a vehicle for finding other readers with which to exchange physical
books for free.

\textbf{Artists.} The art you sell does not have to be free, but the
media describing it does. This is a platform on which artists can
co-create whole books which catalog the art they create and sell or
promote whatever commercial channels they use for that. This is not an
advertising platform. Spam gets deleted. But it is a place where people
can create long form exposition of whatever they produce and place all
that in the context of other creators' art and craft products. As with
the creations of authors, these books can be sold as physically printed
bound volumes from print-on-demand as well as being shared freely
online. We aim to create a coherent whole out of all the art created in
a given physically local community, to the benefit of all.

\textbf{Deep readers.} The knowledge which can only be attained through
reading a lot of books is of great value to a library community. We need
people to put together libraries, to organize content, to edit, and to
add manuscripts which are already available for free but not widely
distributed. We also need people to curate libraries, to figure out
exactly what people can benefit from in a given community. That can only
happen with very active participation by people who read a lot, both
widely and with some depth into various fields. Reading lists are of the
utmost importance.

\textbf{Practitioners of religion/magic/spirituality.} The Magic Books
we co-create with the Pibrary network are living documents. What better
way to transmit wisdom could you ask for? True wisdom does not belong to
anyone. True wisdom can withstand the maelstrom of a chaotic co-editing
process by potentially billions of readers and writers and end up better
than it started. We ask that you consider sharing what you know and what
you have learned from your teachers in this truly free form.

\textbf{Technology creators.} One of the core functions of this
knowledge network is to spread the technical knowledge required for
people to build a new civilization from the waste streams of the
existing one. This requires a whole new way of creating technology,
based on free sharing of knowledge over this network. Ideally this can
be a self-sustaining way to exist in society as a creator of technology.
We can create technology, share detailed documents on how to build it,
and it will come back to us with community additions much better than
what we built. As the network grows and we build more and more truly
free infrastructure (manufacturing, housing, food production, power,
etc.) we can eventually fully support ourselves off of this network. We
will release our creations for free into the network and get back more
free stuff than we put in as the network effect accelerates innovation.

\textbf{Organizers.} What is organizing if not creating new social
networks? Our network creates shared public spaces for shared public
knowledge, and we invite all organizers to use this to create books
which help create their own social networks. This can be a political
group, a union, or an affinity group of any kind. Rather than a flow of
information in a news feed or a ``page'' in some social media platform,
you create books which document the social system you are creating with
a greater depth and permanence. This can include a deeper examination of
your motives and ideology than would be possible in a shorter format.
You can think of this as a manifesto, a constitution, a history, or any
other deep form of knowledge which is the basis of what you're
organizing.

\textbf{Local government.} This network of documents can be used to
create community knowledge which includes the activities of local
government, increasing engagement and effectiveness. One of the main
goals of creating local networks is to bring power away from central
governments and down to the local level. You are responsible for helping
to promote economic vitality in your area, and building free public
network resources fulfills this mission. As with public librarians, we
hope to create partnerships with local governments to get grants for
development of the network in addressing issues of the ``digital
divide''. Creating strong local networks can bring in money and jobs and
people. As we hone the methods of this network we can provide
increasingly tried and tested systems for you to raise money, build
infrastructure, and reap the rewards.

\textbf{Owners of public spaces like shopping centers.} Creating a
locally controlled network which is free for everyone and which enhances
the depth and meaning of a space increases the value of that space in
every way. People who own spaces with a lot of businesses open to the
general public can collaborate with the people of the network to find
business owners and other community members to work with to provide the
network access, host the servers, and pay for the domains and cloud
hosting. We are building new social networks, new communities which do
not currently exist around the shared culture of working toward building
full Trash Magic. This common cause will cause excitement and foot
traffic, increasing wealth generation for all people involved in the
space. As we grow the Trash Factory system, we will need retail outlets
for some of the things we manufacture, to both give away and sell, and
properties with retail can benefit from this added business.

\textbf{Art Gallery Owners.} An art gallery is already a type of
library, as well as a network. It links artists with the public and with
patrons. It links art with the rest of the cultural context in which
that art was created. And finally it is a physical place, the survival
of which depends on how people view its meaning. Building a free book
which represents the meaning of the \emph{place} that is a gallery,
studio or other art space can act to amplify the value of that place,
benefiting the mission of the space.

\textbf{Role Playing Creators} Use the Map Book to create a mixed
reality game in a physical space. Games can create complex layers of
meaning and experience in a physical space. Work with the people who
already inhabit that space as well as other people we draw into the
space to co-create new layers of reality using the medium of the Map
Book. This project might turn out to be the most powerful in the whole
system. Our aim in this work is to create social media which is a hybrid
between the physical space our bodies inhabit and the media which
defines the world our minds inhabit. This work requires imagination and
story telling. If we can engage the imagination of game-creators it can
make these worlds much richer and inspire the kind of actions we need to
get all the other people involved who will make our network a reality.
