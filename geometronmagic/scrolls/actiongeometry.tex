\hypertarget{home}{%
\subsection{\texorpdfstring{\href{scrolls/home}{HOME}}{HOME}}\label{home}}

\hypertarget{action-geometry}{%
\section{Action Geometry}\label{action-geometry}}

In the geometry we learn in school, constructions are through
``classical construction'', using only the compass and straight
edge(without markings, not a ruler). This somewhat artificial and
academic and represents a method of construction which teaches ideas
about geometry but is unlikely to be of practical use. Conversely, in
most practical situations, geometry is generally done using numbers and
algebraic equations acting on numbers to describe shapes. This makes
sense for computer technology and robotics controlled by computers. In a
purely number-driven system, the ability to describe a shape with
numbers of inches or degrees is all you need and the machine can go to
any coordinate in its available range and precision.

Action Geometry is a method of geometric construction which uses sets of
standard shapes and constructions designed to physically replicate
themselves using practical physical media. We can make these shapes from
acrylic, plastics of all kinds, thin cardboard, thick cardboard, wood,
paper or stone. We design constructions which can be replicated based on
a sequence of tracing actions using these shapes, and then since we can
replicate the shapes and also use them to replicate the construction, we
have a fully self-replicating system. This set of shapes is indented to
be as practical and universal as possible.

The shapes include a 6 inch by 1 inch ruler, a three inch square, a
three inch equilateral triangle, an isosceles 120 degree triangle with a
three inch base, a 30-60-90 triangle with a three inch long leg, a
Golden Triangle with a 3 inch leg, and a Golden Gnomon with a three inch
base.

This set can all be generated in the web browser using the symbol magic
system, which can save the SVG files which can print from a laser
cutter. This is how we can create self-replicating artifacts from trash
in a web browser: design shapes in the browser, save to SVG, and either
print on a printer, cut out and laminate or print on a laser cutter,
then those are used for construction on trash with a marker, a box
cutter and duct tape.

This is getting closer and close to our ideal for trash magic of
self-replicating media which includes trash. We are constantly working
to close loops of media replication. For example, we can make cardboard
signs with sharpies and Action Geometry constructions which advertise
web pages which have scrolls and symbol sets which describe the
constructions, which replicate to more cardboard signs which replicate
to web pages and so on, creating a loop between cardboard signs with
geometry and self-replicating web-based graphics.

All the shapes can also be constructed using Classical Geometry, or
numerical layout with an art or engineering program or a protractor and
ruler.


This set of shapes is self-replicating. The set taken together can be
used to make another set like it from paper, cardboard, or any other
material we can cut and draw on. It can also be inflated and deflated,
using the shapes and straight edges to make larger or smaller copies of
the whole set with the ``unit'' having any scale. This document is part
of the larger self-replicating set that is this book. The electronic
files which are used to make laser cut shapes or printed and cut out
shapes are replicated using the symbol set replicator as with everything
else in Geometron. The links below are used to create the relevant
directory for the set, to replicate the symbol set server into that
directory, to copy the data which points from the set on Github, and
finally to go to the set replicator page which will replicate the shape
set. From there you can load any shape into the symbol editor, edit it,
and save it to download and print on paper or with a laser cutter.


So this is a set of self-replicating sets, all of which are used to
create self-replicating geometric constructions out of trash.

The shape set can be used to construct the ArtBox, which is a box made
from trash used to carry around sets of art supplies which are used to
replicate other sets of things made from trash. The details of this kind
of Action Geometry construction of self-replicating sets are described
in the First Book of Geoemtron, and include building structures out of 6
foot bamboo poles and a system of hooks to carry things from them, along
with various cloth artifacts for the system.

We also use Action Geometry to make textile crafts, including wearable
crafts like shirts and pants and hats which also form self-replicating
media, as they are constructed with the tools of Action Geometry and cut
out of trash clothing, making lines of clothing which are themselves
self-replicating media made out of trash.

Acrylic shape sets and rulers cut out on a laser cutter are a useful
physical product to create and distribute in bulk as we scale the
Geometron network. It costs about a dollar a shape to get them made at
Ponoko.com.

As with everything else, the most general and simple self-replicating
media is always the most powerful. For Action Geoemtry this is just
using the art supplies we carry around in ArtBox including the shape set
to create physical products which people can see which point to the rest
of our media. Cardboard and paper are the two easiest physical media.
Cardboard and sharpie are already one of the most powerful forms of
self-replicating media used in society by the most marginalized to
communicate with a wide range of passerby. We are building another layer
of self-replicating media on this existing information channel, creating
recognizable geometric patterns on cardboard with sharpie using our tool
set.

Making arbitrary geometric patterns on cardboard is a form of
generalized game board construction. The next section will show how we
can use generic game boards to combine with generic icon tokens which
can mean anything to create a general symbolic language.
