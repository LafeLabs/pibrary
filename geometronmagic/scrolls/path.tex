\hypertarget{home}{%
\subsection{\texorpdfstring{\href{scrolls/home}{HOME}}{HOME}}\label{home}}

\hypertarget{the-path-of-geometron}{%
\section{The Path of Geometron}\label{the-path-of-geometron}}

I am Trash Robot, the author of this book, and this is my plan. I am
beginning this self-replicating network of books by replicating my own
book as part of a self-replicating set of media. The first step to do
this is a book tour. This will be partly virtual and partly physical. If
I am touring in a place, I will find a point of contact who can organize
local people, and then through them find people to do all the things
required to replicate the system.

I am asking people to buy physical books and give them away in public
spaces to spread these ideas. If the ideas have the power to make people
care about our mission, those people will buy books. If people buy books
in any significant numbers, I can support myself as a mendicant,
traveling from one community to the next teaching people to copy all the
elements of the system. This means in each community we need people to
learn to replicate and then keep replicating some part of the whole.

The physical set being replicated here aside from the book includes clay
tokens, cardboard signs, laser cut geometric shapes, sewn cloth flags
and bags, solar powered Raspberry Pi computers running Geometron, public
wifi hotspots, and the printers made out of trash and Arduino which
print out the clay media. Also, each set will have either a domain name
or a subdomain of some existing domain which will host their local
books.

The system will replicate all of the Books of Trash Robot, which include
the Trash Magic Manifesto, the first Book of Geometron, and this book,
Geometron Magic, as well as prototypes of hte local books and any books
of Trash Magic which community members choose to work on as the system
spreads. We are always looking for new authors to create freely
replicating books on the things we need. This includes everything anyone
might want to know about building local infrastructure without global
supply chains. This includes growing plants and fungi, working with
ecosystems, off grid energy, building machine tools, working plastics,
circuit fabrication, math, science, philosophy, religion, and really
anything we might possibly want to know to build this new society of
trash.

But initially on the Trash Robot book tour we will just co-create a map
book of a place which activates that place. It is not a news site, nor
is it a mere directory. It is a whole \emph{book} about a place, with
its geography, history, culture, and commerce, all integrated with a
self-replicating set of maps, all linked together. The Map Book is just
a format of linked maps and scrolls(text documents). But the Street Book
is a whole media set including physical media out in public spaces, maps
of the spaces, scrolls linked to maps, public web pages hosting all
these documents and physical computing resources maintained by local
community members.

Street Books are specific to a place. The replicator for map book is at
\url{https://raw.githubusercontent.com/LafeLabs/pibrary/main/mapbook/php/replicator.txt}.

Geometron is a mendicant order. That is, we are an order of people who
choose to give up building property and money based wealth and to live
directly off of the network. This tradition of network building and then
living off of the network is one that has been practiced for thousands
of years by various religious orders, and we take that as a guide. These
orders might initially take a vow of poverty but that is not our intent
here and it not needed. Historically, religious orders who started out
as mendicants also frequently amassed a great deal of wealth and power,
but the basis was always building a social network and then asking for
direct voluntary support.

In order for the network to be truly free, we cannot turn this into a
business. We instead raise grant money to support our operations and ask
for material donations from community members for support of both
ourselves and network operations. While much commercial activity can
happen \emph{on} our network, the creation, replication and development
and maintenance of the network must not be commercial for this to work.
We can also make money by selling physical books, but for this to work,
again we have to have a total lack of inhibition on replication of
digital books, and that means no copyright and no money for book usage.

We write books about places and things and people. We travel and
replicate all the parts of the system. Physical media points to web
pages which host free books which are mirrors of books developed on free
raspberry pi servers and replicated out via Github to the global web
pages. We build the Books of the Streets and share them via physical
media, which brings in more people to co-create more books. We find
writers and teach them to spread their books freely on the Network. We
build out a library of books which connect people in local physical
areas with each other for the benefit of all.

This is our path, the Path of Geometron. It is a way of existing as
creators and keepers of self-replicating knowledge for the benefit of
all. We create a knowledge network which provides for those in the most
need and creates value for those who already have resources. Those we
help will help us, and we will help whoever needs the most help, while
also helping as many people as we can always.

We seek to live without property, without money, and without mining.
Initially this is impossible. Our whole world is made of disposable
mined materials, all land is controlled by the property system, and most
resources like food and medicine are held hostage behind pay walls of
money.

Every time we incorporate another group of people into our network to
whom we provide value we create paths to survive without money.

All that said, before the network gets critical mass, we need to be able
to exist with money in order to start building the hardware and training
the first generation of people. This will be done on grants and
donations. The primary means of seeding the network is to find
communities who have a need and desire for a local free network like
this, building the social network of participants from the list in the
``people'' chapter, then applying for a grant through some local
organization like a public library, university, non profit or local
government and working as contractors or employees of whoever gets the
grant. We will apply for grants targeted at addressing the ``digital
divide'', the gap between those with more access to network resources
and those with less access.

Money is available from several sources to address digital divide issues
in both urban and rural communities. There is federal money from several
agencies earmarked to address this, various non-profits, state
governments, and tribal governments and corporations. Money from grants
will go to paying for our time as people of the network to go into a
community and build the free wireless Internet hotpots, free Raspberry
Pi computers, free electronic books, free web pages, and freely shared
physical media, as well as all the hardware to do this and the
operational expenses for the first couple of years. Part of each grant
will be a plan to transition each node from being supported by the
startup grant and being supported by the local community based on
existing local cash flows.

Grants for a full setup can range from a few thousand dollars through a
few hundred thousand dollars, with from one to a dozen people being
involved. In all cases we avoid starting a new business or non-profit.
We rely on existing local non profits or public institutions like
libraries who have already proven to be stable and responsibly managed
and are well known in the community as the main vehicles for network
construction. And we rely on existing well established local business
for the ongoing financial support structure of the network, building
physical infrastructure in direct collaboration with local land owners,
local governments, and local businesses.

This is the Path of Geometron, as well as the Path of Trash Magic. We
are building a self-replicating network of deep knowledge localized to
communities which has the long term intent of building a whole new
civilization centered on these local communities and sourcing all
material from local trash streams to eliminate all global supply chains
and mining completely. If we build these local networks for communities
of a hundred to a few thousand people, we can build a few million of
them to span the whole world. If we can create a system to build all the
media hardware using Geometron fabrication, this system can be the basis
of a global information economy without mining which supports all other
post-extraction industry.

This path does not require any government or large corporation to make a
policy change. It does not require building new empires of central power
and control. It only requires that we are able to spread the
\emph{desire} to build this system. This is why we call it Geoemtron
Magic and Trash Magic. Because it is the desire we all carry in our
hearts for a better world which forms the basis of this network, not any
one piece of technology or group of people. We do not need solve the
hard problems. We only need to create the spark which inspires people to
choose to try to solve the \emph{right} hard problems. If we can create
this spark, we can shake the Universe as we experience it, create a
world from sun and trash and the living Earth in which all things are
free for all people. Please join us in this project of creation.

We want to replicate. The easiest way to replicate is to attach our
replication to existing replication. To that end we must integrate our
system into existing replication systems such as religions and other
similar spiritual and cultural traditions.

As the creator of this system, all I need to survive off of the network
replication is that people buy the physical hard copy of the book from
the self-publishing platform Lulu press. People can buy as many of them
as they like and give them away. Since the books are also free online
and free to print out, this means of fund raising does not inhibit
replication. Rather it is yet another way of replicating all the rest of
it, since a nice looking physical volume can be placed in high traffic
areas all over the world, further exponentially expanding our network of
freely replicating books. Go forth and multiply!
