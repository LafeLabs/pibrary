\hypertarget{home}{%
\subsection{\texorpdfstring{\href{scrolls/home}{HOME}}{HOME}}\label{home}}

\hypertarget{the-pibrary}{%
\section{The Pibrary}\label{the-pibrary}}

The Pibrary is a network of free books distributed using the Raspberry
Pi, a very cheap open source computer designed primarily for educational
use. The goal of the Pibrary is to be an extension of the public library
system into more public spaces and with more free and more local books.

The Pibrary represents a fully free network, meaning everything is free
of copyrights or other intellectual property, everything is available in
a public space for use by whoever wants to use it, and everything can be
freely replicated by other people in other places.

The Pibrary is centered on public spaces, meaning spaces anyone can get
access to without any restrictions such as public parks or public
streets or any other outdoor space which we do not restrict access to
based on money. This can include private property as long as the owners
of the property are willing to create a truly open space which welcomes
all people regardless of social or economic status on a 24/7 basis.

The free computing element of the Pibrary consists of mobile raspberry
pi computers with portable battery and solar panel as a free community
computing resource for use in public spaces by whoever has the greatest
need for free access to information. This is to be an extension of other
mutual aid projects like Food Not Bombs which provide basic goods and
services for free to the most marginalized people in public spaces. The
solar panels and batteries can be used to power or charge mobile
devices, a critical infrastructure need for unhoused or traveling
people. The Raspberry Pis are installed with no private data, no logins,
no passwords, and are intended to be used that way so that they can be
safely shared. The only purpose of the Pi is to access the Internet for
free, just like a free public computer in a public library.

The basic Raspberry Pi mobile terminal setup consists of the Pi board
which is about the size of a deck of cards and costs about \$60, the SD
card the operating system is installed on, a keyboard, a mouse, a small
mobile screen, a 12 volt lead acid battery, and a solar panel and
charger. The whole system costs about \$400 and can all be purchased
online from many vendors. In order to charge devices the kit must also
have a 12 volt USB power hub to break out the battery power for charging
USB devices.

The Raspberry Pi is also used as a home web server for creating, editing
and sharing the self-replicating documents of the Geometron system. To
run a Pi at home we need a lot less infrastructure. The keyboard and
mouse and a standard TV or computer screen can be used once to set it
up, it can be powered off of a wall plug, and then run ``headless'' with
no screen or peripherals, accessed over entirely over the network. This
server can be accessed by people anywhere in the world by using port
forwarding over the home router or router in a public space to connect
it to the outside Internet.

Internet access is provided for free in public spaces by wireless
hotspots with clearly posted log on information. We can beam Internet
into public spaces with directional antennas and wireless network
extenders. All of this physical infrastructure is provided for free by
donors from the local community. It is a public resource.

The primary purpose of all the media hosted on the Pibrary system is to
create a free library of books which contain all the knowledge required
to build a society based on the principles of the last chapter: built
from trash and powered from the sun, wind and water. This will include
science, technology, history, culture, commerce, and any other deep
knowledge about and for a local community. All of these books consist of
collections of documents which replicate freely from one server to
another without any restrictions. Each copy of each book can also be
edited, deleted, or moved around on any server by anyone at any time.

The Pibrary is a network of free self-replicating books. This is not a
network of users. There is no private data, there are no users, no
logins, no passwords, no encryption, and no databases. While we are
creating a vast universe of documents, each Pibrary will have a
collection which is limited based on a focus of immediate interest for a
relatively small community with a shared purpose.

Part of the infrastructure of the Pibrary is domain names for public use
which can host copies of all the documents in our system. All of these
public web pages, like the Raspberry Pi servers, have no private data,
no users, and no databases. Anyone can copy files onto them and off of
them, or delete them at will. Pages can be forked down to make more
libraries with more books and more libraries inside libraries. Whole
forks can be deleted by anyone at any time. Our resilience against
deleting is to constantly copy books to many places again and again. We
are building media which behaves as a living organism, replicating,
dying, and evolving as part of our culture as humans.

The domain names for these public web pages are selected to correspond
to public spaces like streets, parks or bodies of water, with top level
domains other than .com so that they are non commercial and not
personal. Some volunteer from the community can buy the domain and pay
for commercial cloud hosting at some standard web hosting vendor, and
install the system on that server. It generally costs about 10 dollars a
year for the domain and another 10 dollars a month or so for the
hosting. As long as our network is providing significant community
benefit this should be a relatively minor cost to get paid for by a
volunteer.

We create physical flags to display in public spaces which point to the
domains which host the books. These flags are created by sewing solid
rainbow colored felt letters in a block font onto a black cloth
background about 3 feet by 3 feet square. Flags, like everything else,
are meant to be copied widely and displayed publicly. Flags fly in
physical spaces which are represented by domains which host books we
create, edit and replicate using the network of free Raspberry Pi
computers.

Another powerful tool in our network of self-replicating books is
Github. Github is a company which provides free(free for open source and
that includes everything we do) hosting of documents which can be copied
from anywhere on the Web. We can create private instances of Geometron
servers on home personal computers which have local web servers set up
which only run on that machine. These are used to edit local copies of
the whole system including any books we want to save. As these are
replicated and edited, they can be ``pushed'' to Github with Github
Desktop, a GUI app. Using Github to move books around provides a backup
where if servers are all wiped out the data can't be edited without
access to a personal Github account which is based on Github's security.
Also Github has enough bandwidth and protection against surges in
bandwidth that it can be a source for replication to many servers all at
once. Anyone anywhere in the world can copy whole libraries of books
with simple clicks in their web browser to their personal Github
repositories, then push it out to the public and copy from there to any
other server. This network of potentially millions of Github accounts
and millions of Raspberry Pi's and millions of domain names can be
constantly supporting a free flow of replication of books from server to
server across the globe.

The format of ``books'' on the Pibrary is the ``magic book'' described
in the next chapter. This library of free books can form the basis of a
social network which provides the same benefits of modern networking
applications but with direct community control. Books can be created to
document all commerce, organized by the people engaging in that
commerce. The same efficiency improvements which are currently monetized
by Silicon Valley can then be kept in local communities, which brings in
enough wealth to materially support the people building the network. As
the amount of wealth generated by the network increases we will direct
all excess to those in the local community with the most need.

The use of the Network to direct resources to those in the most need is
mutual aid. The network helps people and those people help the network
by representing it in public, sharing the information with as many
people as possible. This applies to everyone. As the network generates
more wealth it should be possible to eliminate poverty in very localized
areas covered by the network. As this happens we can use the network to
build more and more industries up using the Trash Factories and this can
amplify the process. Network value in commercial activity funds
industrial value which funds more network expansion and so on. As the
network grows and we can support more people, those people can solve
harder and harder problems and scale up what we can make in the Trash
Factories more and more. As this knowledge is generated, it will all be
synthesized into more free self-replicating books which are published
onto the Pibrary network. So our manufacturing supports growth of the
network, but the network is also supporting the growth of manufacturing
by replicating all the knowledge required to copy our processes.

The Pibrary creates and enhances public spaces. Selecting the right
physical space to inhabit for this is one of the most important parts of
building a successful network. We need to choose spaces that have the
absolute maximum possible intersection of people. We must ask the
question: if a place is about 10 yards across, what place in a given
area a couple miles across has the widest range of people crossing it in
any given day? Of places like this, what is the most freely accessible?
We must evaluate accessibility based on sidewalks, access by car, access
by public transit, by bike, wheelchair, or any other means of access
relevant to the local community. But we must also consider accessibility
in terms of it being legal to be there, there being adequate restroom
facilities nearby, places to rest or work or sleep, shade or other
shelter, and a generally welcoming culture.

The public space being activated by a Pibrary does not need to always
have a Raspberry Pi or solar panel. It can just have a flag or sign or
markers which point to the domain which has the copies of the books
maintained about that place. It can even be invisible, with a known
domain being used by people about a place without any obvious
infrastructure in that place other than the place itself.

The Raspberry Pi can also serve as a monitor for the environment,
measuring aspects of the water, air, soil, living things and anything
else of interest to the community. In rural areas, sequences of wireless
network repeaters on off-grid power can go along rivers and streams with
local Raspberry Pi's with sensors measuring water properties and
delivering that information via the web to the rest of the network. This
can put the land, life, and water itself onto the network and connect
all of us humans on the network more intimately with these systems.
